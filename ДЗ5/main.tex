\documentclass[a4paper]{article}
\usepackage{setspace}
\usepackage[T2A]{fontenc} %
\usepackage[utf8]{inputenc} % подключение русского языка
\usepackage[russian]{babel} %
\usepackage[12pt]{extsizes}
\usepackage{mathtools}
\usepackage{graphicx}
\usepackage{fancyhdr}
\usepackage{amssymb}
\usepackage{amsmath, amsfonts, amssymb, amsthm, mathtools}
\usepackage{tikz}

\usetikzlibrary{positioning}
\setstretch{1.3}

\newcommand{\mat}[1]{\begin{pmatrix} #1 \end{pmatrix}}
\renewcommand{\det}[1]{\begin{vmatrix} #1 \end{vmatrix}}
\renewcommand{\f}[2]{\frac{#1}{#2}}
\newcommand{\dspace}{\space\space}
\newcommand{\s}[2]{\sum\limits_{#1}^{#2}}
\newcommand{\mul}[2]{\prod_{#1}^{#2}}
\newcommand{\sq}[1]{\left[ {#1} \right]}
\newcommand{\gath}[1]{\left[ \begin{array}{@{}l@{}} #1 \end{array} \right.}
\newcommand{\case}[1]{\begin{cases} #1 \end{cases}}
\newcommand{\ts}{\text{\space}}
\newcommand{\lm}[1]{\underset{#1}{\lim}}
\newcommand{\suplm}[1]{\underset{#1}{\overline{\lim}}}
\newcommand{\inflm}[1]{\underset{#1}{\underline{\lim}}}

\renewcommand{\phi}{\varphi}
\newcommand{\lr}{\Leftrightarrow}
\renewcommand{\r}{\Rightarrow}
\newcommand{\rr}{\rightarrow}
\renewcommand{\geq}{\geqslant}
\renewcommand{\leq}{\leqslant}
\newcommand{\RR}{\mathbb{R}}
\newcommand{\CC}{\mathbb{C}}
\newcommand{\QQ}{\mathbb{Q}}
\newcommand{\ZZ}{\mathbb{Z}}
\newcommand{\VV}{\mathbb{V}}
\newcommand{\NN}{\mathbb{N}}
\newcommand{\OO}{\underline{O}}
\newcommand{\oo}{\overline{o}}
\newcommand{\divides}{\;|\;}

\DeclarePairedDelimiter\abs{\lvert}{\rvert} %
\makeatletter                               % \abs{}
\let\oldabs\abs                             %
\def\abs{\@ifstar{\oldabs}{\oldabs*}}       %

\begin{document}

\section*{Домашнее задание на 07.02 (Теория чисел)}
 {\large Емельянов Владимир, ПМИ гр №247}\\\\
\begin{enumerate}
    \item[\textbf{№1}]
    Рассмотрим сравнение \( x^3 - 17x^2 - 7x + 11 \equiv 0 \pmod{54} \).  

    Модуль \(54 = 2 \cdot 3^3\). Решим систему сравнений по модулям 2 и 27.
    \begin{enumerate}
        \item[1)]По модулю 2:
        
        Подстановка \(x = 0\) и \(x = 1\):  
        \[x \equiv 0 \pmod{2}: 1 \not\equiv 0 \pmod{2}\]
        \[x \equiv 1 \pmod{2}: 0 \equiv 0 \pmod{2}\]
        Решение: \(x \equiv 1 \pmod{2}\)

        \item[2)]По модулю 27:  
        
        Сначала решаем по модулю 3:  
        \[x^3 - 2x^2 + 2x + 2 \equiv 0 \pmod{3}\]  
        Проверка \(x = 0, 1, 2\):  

        $$\text{\(x \equiv 1 \pmod{3}\) и \(x \equiv 2 \pmod{3}\) — решения}$$

        Подъем решений до модуля 9: 

        Для \(x \equiv 1 \pmod{3}\): 
        \[x \equiv 4 \pmod{9}\]  
        Для \(x \equiv 2 \pmod{3}\): 
        \[x \equiv 2 \pmod{9}\]  

        Подъем до модуля 27:  

        \(x \equiv 4 \pmod{9}\) даёт \[x \equiv 13 \pmod{27}\]
        \(x \equiv 2 \pmod{9}\) не даёт решений. 
    \end{enumerate}

    Объединение решений: 
    \[
    \begin{cases}
    x \equiv 1 \pmod{2}, \\
    x \equiv 13 \pmod{27}.
    \end{cases}
    \implies x \equiv 13 \pmod{54}\]  

    \textbf{Ответ: } $x \equiv 13 \pmod{54}$\\

    \item[\textbf{№2}]
    Рассмотрим сравнение 
    \[x^2 - 25 \equiv 0 \pmod{16^{19} \cdot 19^{91} \cdot 91^{16}}\] 
    Модуль раскладывается: 
    \[N = 2^{76} \cdot 19^{91} \cdot 7^{16} \cdot 13^{16}\] 
    
    Количество решений равно произведению количеств решений по каждому 
    простому множителю:
    \begin{enumerate}
        \item[1)]Для \(2^{76}\):
        
        Уравнение \(x^2 \equiv 25 \pmod{2^{76}}\). Так как \(25 \equiv 1 \pmod{8}\), 
        количество решений — 4.
        \item[2)]Для \(19^{91}\), \(7^{16}\), \(13^{16}\): 

        Уравнение \(x^2 \equiv 25 \pmod{p^k}\) имеет два решения 
        (\(x \equiv 5\) и \(x \equiv -5 \pmod{p^k}\)), так как \(5 \not\equiv -5 \pmod{p^k}\)
    \end{enumerate}
    Итоговое количество решений:  
    \[
    4 \cdot 2 \cdot 2 \cdot 2 = 32
    \]
    \textbf{Ответ: } $32$

    \item[\textbf{№3}]Рассмотрим сравнение \( x^2 \equiv y^2 \pmod{p} \), где \( p \)
     — нечётное простое число. Перепишем его в виде:
    \[
    (x - y)(x + y) \equiv 0 \pmod{p}.
    \]
    Так как $p$ - простое, то:
    $$\gath{ 
        x \equiv y \pmod{p} \\
        x \equiv -y \pmod{p} 
    }$$
    \begin{enumerate}
        \item[1)]Количество пар для \( x \equiv y \pmod{p}\):
    
        Для каждого \( x \in \{0, 1, 2, ..., p-1\} \) существует ровно один 
        \( y \equiv x \pmod{p} \). Таким образом, количество таких пар равно \( p \).
    
        \item[2)]Количество пар для \( x \equiv -y \pmod{p}\): 
    
        Для каждого \( x \in \{0, 1, 2, ..., p-1\} \) существует ровно один 
        \( y \equiv -x \pmod{p} \). Количество таких пар также равно \( p \).
    \end{enumerate}

    Пара \( (0, 0) \) удовлетворяет обоим условиям. Таким образом, она учтена дважды.
    Чтобы получить количество уникальных решений, вычтем 1 повторяющуюся пару.

    Итоговое количество решений:
    \[
    p + p - 1 = 2p - 1.
    \]
    Что и требовалось доказать.

    \item[\textbf{№4}]Рассмотрим сравнение 
    \[(x^2 - ab)(x^2 - bc)(x^2 - ac) \equiv 0 \pmod{p}\]
    Для его разрешимости необходимо, чтобы хотя бы одно из уравнений:
    \[x^2 \equiv ab \pmod{p}, \quad x^2 \equiv bc \pmod{p},
     \quad x^2 \equiv ac \pmod{p}\]

    имело решение. Рассмотрим два случая.
    \begin{enumerate}
        \item[1)]Случай 1: 
        
        Хотя бы одно из чисел \(a, b, c\) делится на \(p\).

        Пусть, например, \(a \equiv 0 \pmod{p}\). Тогда:
        \[ab \equiv 0 \pmod{p}\]
        \[ac \equiv 0 \pmod{p}\]
        
        Уравнения \(x^2 \equiv 0 \pmod{p}\) имеют решение \(x \equiv 0 \pmod{p}\).
        Таким образом, в этом случае сравнение разрешимо.

        \item[2)]Случай 2:
        
        Все числа \(a, b, c\) не делятся на \(p\).

        В этом случае \(a, b, c\) обратимы, а значит и \(ab, bc, ac\) обратимы. 
        Покажем, что хотя бы один из \[ab, bc, ac\] является квадратичным вычетом.

        Предположим противное: все три элемента \(ab, bc, ac\) — 
        квадратичные невычеты. Тогда их произведение:
        \[
        (ab)(bc)(ac) = a^2 b^2 c^2 = (abc)^2
        \]
        Квадрат любого элемента является квадратичным вычетом, 
        поэтому \((abc)^2\) — вычет. Однако произведение трёх невычетов вычисляется как:
        \[
        \text{невычет} \cdot \text{невычет} \cdot \text{невычет} = (\text{вычет}) \cdot \text{невычет} = \text{невычет}.
        \]
        Получаем противоречие: \((abc)^2\) одновременно является вычетом и невычетом. 
        Следовательно, предположение неверно, и хотя бы один из элементов \(ab, bc, ac\)
         — квадратичный вычет. Соответствующее уравнение 
         \(x^2 \equiv \text{вычет} \pmod{p}\) имеет решение.\\

    \end{enumerate}
    В обоих случаях сравнение \((x^2 - ab)(x^2 - bc)(x^2 - ac) \equiv 0 \pmod{p}\)
    разрешимо. Таким образом, утверждение доказано для любого простого \(p\) и любых
    \(a, b, c \in \mathbb{Z}\).\\

    \item[\textbf{№5}]\begin{enumerate}
        \item[а)]\(\sum_{x=0}^{58} \left( \frac{15x+79}{59} \right)\)
        
        Символ Лежандра \(\left( \frac{15x+79}{59} \right)\) 
        преобразуется к \(\left( \frac{15x+20}{59} \right)\), 
        так как \(79 \equiv 20 \pmod{59}\).  

        Коэффициент 15 взаимно прост с 59, поэтому \(15x + 20\) 
        пробегает все вычеты по модулю 59 при \(x = 0, 1, \dots, 58\).

        Сумма символов Лежандра по всем \(a \pmod{59}\) (включая 0) равна 0. Это следует из того, что количество квадратичных вычетов и невычетов одинаково, и их вклады взаимно сокращаются.  

        \textbf{Ответ: } $0$\\

        \item[б)]\(\sum_{x=0}^{57} \left( \frac{15x+79}{59} \right)\)
        
        Так как верхний предел \(x = 57\), а при \(x = 58\) выражение \(15x + 20\)
         даёт \(5 \pmod{59}\), нужно посчитать $\left( \frac{5}{59} \right)$.
        \[
        \left( \frac{5}{59} \right) = 1 \text{ т.к. \(5 \equiv 64 \equiv 8^2 \pmod{59}\)}
        \]  
        Получается, что:
        \[\sum_{x=0}^{57} \left( \frac{15x+79}{59} \right) =
        \sum_{x=0}^{58}\left( \frac{15x+79}{59} \right) - \left( \frac{5}{59} \right) = 
        0-1 = -1\]
        \textbf{Ответ:} $-1$
    \end{enumerate}
\end{enumerate}
\end{document}