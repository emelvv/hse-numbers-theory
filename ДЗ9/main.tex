\documentclass[a4paper]{article}
\usepackage{setspace}
\usepackage[T2A]{fontenc} %
\usepackage[utf8]{inputenc} % подключение русского языка
\usepackage[russian]{babel} %
\usepackage[12pt]{extsizes}
\usepackage{mathtools}
\usepackage{graphicx}
\usepackage{fancyhdr}
\usepackage{amssymb}
\usepackage{amsmath, amsfonts, amssymb, amsthm, mathtools}
\usepackage{tikz}

\usetikzlibrary{positioning}
\setstretch{1.3}

\newcommand{\mat}[1]{\begin{pmatrix} #1 \end{pmatrix}}
\renewcommand{\det}[1]{\begin{vmatrix} #1 \end{vmatrix}}
\renewcommand{\f}[2]{\frac{#1}{#2}}
\newcommand{\dspace}{\space\space}
\newcommand{\s}[2]{\sum\limits_{#1}^{#2}}
\newcommand{\mul}[2]{\prod_{#1}^{#2}}
\newcommand{\sq}[1]{\left[ {#1} \right]}
\newcommand{\gath}[1]{\left[ \begin{array}{@{}l@{}} #1 \end{array} \right.}
\newcommand{\case}[1]{\begin{cases} #1 \end{cases}}
\newcommand{\ts}{\text{\space}}
\newcommand{\lm}[1]{\underset{#1}{\lim}}
\newcommand{\suplm}[1]{\underset{#1}{\overline{\lim}}}
\newcommand{\inflm}[1]{\underset{#1}{\underline{\lim}}}

\renewcommand{\phi}{\varphi}
\newcommand{\lr}{\Leftrightarrow}
\renewcommand{\r}{\Rightarrow}
\newcommand{\rr}{\rightarrow}
\renewcommand{\geq}{\geqslant}
\renewcommand{\leq}{\leqslant}
\newcommand{\RR}{\mathbb{R}}
\newcommand{\CC}{\mathbb{C}}
\newcommand{\QQ}{\mathbb{Q}}
\newcommand{\ZZ}{\mathbb{Z}}
\newcommand{\VV}{\mathbb{V}}
\newcommand{\NN}{\mathbb{N}}
\newcommand{\OO}{\underline{O}}
\newcommand{\oo}{\overline{o}}
\newcommand{\divides}{\;|\;}
\newcommand{\leg}[2]{\left(\f{#1}{#2}\right)}

\DeclarePairedDelimiter\abs{\lvert}{\rvert} %
\makeatletter                               % \abs{}
\let\oldabs\abs                             %
\def\abs{\@ifstar{\oldabs}{\oldabs*}}       %

\begin{document}

\section*{Домашнее задание на 14.03 (Теория чисел)}
 {\large Емельянов Владимир, ПМИ гр №247}\\\\
\begin{enumerate}
    \item[\textbf{№1}]Найдём $\phi(22)$:
    $$\phi(22) = 10$$
    $g$ - первообразный корень по модулю $22$ тогда и только тогда, когда:
    $$\forall q \divides 10: \quad  g^{\f{10}{q}} \not \equiv 1 \pmod{22}$$
    То есть:
    $$\case{
        g^{5} \not \equiv 1 \pmod{ 22}\\
        g^{2} \not \equiv 1 \pmod{22}\\
    }$$
    Подставим каждое значение из приведённой системы вычетов:
    $$g \in \{\pm{1}, \pm{3}, \pm{5}, \pm{7}, \pm{9}\}$$
    Таким образом, подходят только:
    $$g \in \{-3, -5, 7, -9\}$$
    Они и являются первообразными корнями.

    \textbf{Ответ:} $-3, -5, 7, -9$\\

    \item[\textbf{№2}]Найдём первообразный корень по модулю $242 = 2\cdot 11^2$.
    \begin{enumerate}
        \item[1)] Для начала найдём первообразный корень по модулю $11$:
        
        $g$ - первообразный корень по модулю $11$ тогда и только тогда, когда:
        $$\forall q \divides 10: \quad  g^{\f{10}{q}} \not \equiv 1 \pmod{11}$$
        То есть:
        $$\case{
            g^{5} \not \equiv 1 \pmod{11}\\
            g^{2} \not \equiv 1 \pmod{11}\\
        }$$
        Найдём первое удовлетворяющее условию значение из набора:
        $$g \in \{1, 3 \dots, 9\}$$
        Таким образом:
        $$g=7 \text{ - первообразный корень}$$

        \item[2)] $g_1$ -первообразный корень по модулю $11^\alpha$ если:
        $$\case{
            g_1 = g \pmod{11}\\
            g_1^{10} \not \equiv 1 \pmod{121}
        }$$

        $g_1$ лежит в наборе:
        $$g_1 \in \{7, 18\}$$
        Проверим $7$:
        $$7^{10} \not \equiv 1 \pmod{121}$$
        Это верно, следовательно:
        $$7 \text{ --- п.к. по модулю } 11^\alpha$$
        В частности:
        $$7 \text{ --- п.к. по модулю } 121$$

        \item[3)] $g_2$ -первообразный корень по модулю $2\cdot 11^\alpha$ если:
        $$\case{
            (g_2, 2) = 1\\
            g_2 \equiv g_1 \pmod{121}
        }$$
        Это выполняется для $$g_2 = 7$$

        Следовательно, 
        $$7 \text{ --- п.к. по модулю } 2\cdot11^\alpha$$
        В частности:
        $$7 \text{ --- п.к. по модулю } 242$$
    \end{enumerate}
    \textbf{Ответ:} $ 7 $\\

    \item[\textbf{№3}]Найдём $\phi(5)$:
    $$\phi(5) = 4$$
    $g$ - первообразный корень по модулю $5$ тогда и только тогда, когда:
    $$\forall q \divides 4: \quad  g^{\f{4}{q}} \not \equiv 1 \pmod{5}$$
    То есть:
    $$g^{2} \not \equiv 1 \pmod{ 5}\\$$

    Подставим каждое значение из приведённой системы вычетов:
    $$g \in \{1, 2, 3, 4\}$$
    Таким образом, подходят только:
    $$g \in \{2, 3\}$$
    Они и являются первообразными корнями по модулю $5$.

    то есть:
    $$2\pmod{5} \quad \text{ и }\quad  3\pmod{5} \text{ --- п.к.}$$

    Следовательно под условие пункта a) подходят числа из набора:
    $$g' \in \{2, 7, 12, 17, 22, 3, 8, 13, 18, 23\}$$

    Чтобы они не являлись первообразными корнями по модулю 25 для каждого $g'$ должно выполняться:
    $$\exists q \divides 16: \quad  g'^{\f{16}{q}} \equiv 1 \pmod{25}$$
    То есть:
    $$g'^{8}  \equiv 1 \pmod{25}\\$$
    Подставив каждое число, получается, что подходят только:
    $$\{7, 18\}$$
    \textbf{Ответ: } $7, 18$\\


    \item[\textbf{№4}]Пусть $g$ - первообразный корень по модулю 29. Тогда $x$ можно представить как 
    $$g^k,\quad  k\in \{0, 1, \dots 27\} \text{ по модулю 28}$$
    Найдём количество решений сравнения:
    $$g^{21k} \equiv 1 \pmod{29}$$
    Чтобы сравнение выполнялось, нужно, чтобы:
    $$21k \equiv 0 \pmod{28} \lr 3k \equiv 0 \pmod{4} \lr k \equiv 0 \pmod{4} $$
    $$\lr k = 4m \quad m \in \ZZ$$
    Следовательно, так как $k\in \{0, 1, \dots 27\}$, то всего решений:
    $$0 \leq 4m < 28 \lr 0 \leq m < 7 \implies \text{7 решений}$$
    \textbf{Ответ:} $7$\\

    \item[\textbf{№5}]Докажем, что число Ферма \( f_n = 2^{2^n} + 1 \) простое при условии 
    \[
    3^{(f_n-1)/2} \equiv -1 \pmod{f_n},
    \]
    Показатель числа 3 по модулю \( f_n \) — это наименьшее натуральное число \( k \), такое что 
    \[
    3^k \equiv 1 \pmod{f_n}.
    \]
    Из условия \( 3^{(f_n-1)/2} \equiv -1 \pmod{f_n} \) следует, что:
   
    Возведя обе части в квадрат: 
    \[
    3^{f_n-1} \equiv 1 \pmod{f_n}.
    \]
    Значит, показатель числа 3 делит \( f_n - 1 \).
    
    Однако 
    \[ 3^{(f_n-1)/2} \not\equiv 1 \pmod{f_n} \]
    Поэтому показатель не делит \( \frac{f_n-1}{2} \)
    
    Следовательно, показатель числа 3 равен \( f_n - 1 \).

    \begin{itemize}
        \item Если \( f_n \) простое, то по малой теореме Ферма 
        \[
        3^{f_n-1} \equiv 1 \pmod{f_n},
        \]
        и показатель числа 3 может достигать \( f_n - 1 \).
        \item Если \( f_n \) составное, то функция Эйлера \( \varphi(f_n) \) будет меньше \( f_n - 1 \), так как у составного числа есть делители, отличные от 1 и самого числа.
        \item Показатель числа 3 должен делить \( \varphi(f_n) \). Но если \( \varphi(f_n) < f_n - 1 \), то показатель \( f_n - 1 \) не может быть делителем \( \varphi(f_n) \). Это противоречие.
    \end{itemize}

    Единственный случай, когда показатель числа 3 равен \( f_n - 1 \), возможен только если \( f_n \) простое. Таким образом, условие 
    \[
    3^{(f_n-1)/2} \equiv -1 \pmod{f_n}
    \]
    гарантирует простоту числа Ферма \( f_n \).

\end{enumerate}
\end{document}