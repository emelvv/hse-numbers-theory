\documentclass[a4paper]{article}
\usepackage{setspace}
\usepackage[T2A]{fontenc} %
\usepackage[utf8]{inputenc} % подключение русского языка
\usepackage[russian]{babel} %
\usepackage[12pt]{extsizes}
\usepackage{mathtools}
\usepackage{graphicx}
\usepackage{fancyhdr}
\usepackage{amssymb}
\usepackage{amsmath, amsfonts, amssymb, amsthm, mathtools}
\usepackage{tikz}

\usetikzlibrary{positioning}
\setstretch{1.3}

\newcommand{\mat}[1]{\begin{pmatrix} #1 \end{pmatrix}}
\renewcommand{\det}[1]{\begin{vmatrix} #1 \end{vmatrix}}
\renewcommand{\f}[2]{\frac{#1}{#2}}
\newcommand{\dspace}{\space\space}
\newcommand{\s}[2]{\sum\limits_{#1}^{#2}}
\newcommand{\mul}[2]{\prod_{#1}^{#2}}
\newcommand{\sq}[1]{\left[ {#1} \right]}
\newcommand{\gath}[1]{\left[ \begin{array}{@{}l@{}} #1 \end{array} \right.}
\newcommand{\case}[1]{\begin{cases} #1 \end{cases}}
\newcommand{\ts}{\text{\space}}
\newcommand{\lm}[1]{\underset{#1}{\lim}}
\newcommand{\suplm}[1]{\underset{#1}{\overline{\lim}}}
\newcommand{\inflm}[1]{\underset{#1}{\underline{\lim}}}

\renewcommand{\phi}{\varphi}
\newcommand{\lr}{\Leftrightarrow}
\renewcommand{\r}{\Rightarrow}
\newcommand{\rr}{\rightarrow}
\renewcommand{\geq}{\geqslant}
\renewcommand{\leq}{\leqslant}
\newcommand{\RR}{\mathbb{R}}
\newcommand{\CC}{\mathbb{C}}
\newcommand{\QQ}{\mathbb{Q}}
\newcommand{\ZZ}{\mathbb{Z}}
\newcommand{\VV}{\mathbb{V}}
\newcommand{\NN}{\mathbb{N}}
\newcommand{\OO}{\underline{O}}
\newcommand{\oo}{\overline{o}}
\newcommand{\divides}{\;|\;}

\DeclarePairedDelimiter\abs{\lvert}{\rvert} %
\makeatletter                               % \abs{}
\let\oldabs\abs                             %
\def\abs{\@ifstar{\oldabs}{\oldabs*}}       %

\begin{document}

\section*{Домашнее задание на 24.01 (Теория чисел)}
 {\large Емельянов Владимир, ПМИ гр №247}\\\\
\begin{enumerate}
    \item[\textbf{№1}]Докажем равенства:
    \begin{enumerate}
        \item[а)]$(a, b) = (a+b, [a,b])$
        Пусть:
        $$a = p_1^{\alpha_1}\cdot...\cdot p_n^{\alpha_n}, \quad b = p_1^{\beta_1}\cdot ... \cdot p_n^{\beta_n}$$
        Пользуясь тем, что:
        $$(a, b) = p_1^{\min(\alpha_1, \beta_1)}\cdot \dots \cdot p_n^{\min(\alpha_n, \beta_n)}$$
        $$[a, b] = p_1^{\max(\alpha_1, \beta_1)}\cdot \dots \cdot p_n^{\max(\alpha_n, \beta_n)}$$
        Перепишем искомое равенство:
        $$p_i^{\min(\alpha_i, \beta_i)} = (p_i^{\alpha_i}+p_i^{\beta_i}, p_i^{\max(\alpha_i, \beta_i)})$$
        Без ограничения общности будем считать, что:
        $$\alpha_i \leq \beta_i \quad \forall i$$
        Тогда получим:
        $$p_i^{\alpha_i} = (p_i^{\alpha_i}+p_i^{\beta_i}, p_i^{\beta_i}) \implies p_i^{\alpha_i} = (p_i^{\alpha_i}(1+p_i^{\beta_i-\alpha_i}), p_i^{\beta_i}) \lr$$
        Следовательно, так как $1+p_i^{\beta_i-\alpha_i}$ не делится на $p_i$, то:
        $$\lr p_i^{\alpha_i} = p_i^{\alpha_i} \text{ - верно}$$

        \item[б)]$\f{(a,b)(b,c)(c,a)}{(a,b,c)^2} = \f{[a, b][b, c][c, a]}{[a, b, c]^2}$
        
        Пусть:
        $$a = p_1^{\alpha_1}\cdot...\cdot p_n^{\alpha_n}, \quad b = p_1^{\beta_1}\cdot ... \cdot p_n^{\beta_n}, \quad c = p_1^{\gamma_1}\cdot ... \cdot p_n^{\gamma_n}$$
        Пользуясь тем, что:
        $$(a, b) = p_1^{\min(\alpha_1, \beta_1)}\cdot \dots \cdot p_n^{\min(\alpha_n, \beta_n)}$$
        $$[a, b] = p_1^{\max(\alpha_1, \beta_1)}\cdot \dots \cdot p_n^{\max(\alpha_n, \beta_n)}$$
        Искомое равенство можно переписать как:
        $$\f{\min(\alpha_i,\beta_i)\min(\beta_i,\gamma_i)\min(\gamma_i,\alpha_i)}{\min(\alpha_i,\beta_i,\gamma_i)^2} = \f{\max[\alpha_i, \beta_i]\max[\beta_i, \gamma_i]\max[\gamma_i, \alpha_i]}{\max[\alpha_i, \beta_i, \gamma_i]^2} \quad \forall i$$
        Без ограничения общности будем считать, что:
        $$\alpha_i \leq \beta_i \leq \gamma_i \quad \forall i$$
        Тогда:
        $$\f{\min(\alpha_i,\beta_i)\min(\beta_i,\gamma_i)\min(\gamma_i,\alpha_i)}{\min(\alpha_i,\beta_i,\gamma_i)^2} = \f{\max[\alpha_i, \beta_i]\max[\beta_i, \gamma_i]\max[\gamma_i, \alpha_i]}{\max[\alpha_i, \beta_i, \gamma_i]^2} \lr $$
        $$\lr \f{\alpha_i\beta_i\alpha_i}{\alpha_i^2} = \f{\beta_i\gamma_i\gamma_i}{\gamma_i^2} \lr \beta_i = \beta_i \text{ - верно}$$
    \end{enumerate}
    
    \item[\textbf{№2}]Чтобы узнать, каким количеством нулей оканчивается число $1000!$, узнаем сколько в нём делителей равных 2 и 5. 
    
    Для этого воспользуемся формулой:
    $$\nu_{p}(n!) = \left[\f{n}{p}\right]+\left[\f{n}{p^2}\right]+ \dots $$
    Получается:
    $$\nu_{2}(1000!) = \left[\f{1000}{2}\right]+\left[\f{1000}{4}\right]+ \dots + \left[\f{1000}{512}\right] = 994$$
    $$\nu_{5}(1000!) = \left[\f{1000}{5}\right]+\left[\f{1000}{25}\right]+ \dots + \left[\f{1000}{625}\right] = 249$$
    Следовательно, делителей равных 10 у числа $1000!$:
    $$\min(249, 994) = 249$$
    \textbf{Ответ: } $249$

    \item[\textbf{№3}]Нам известно, что $n = p_1^{\alpha_1}\cdot \dots \cdot p_s^{\alpha_s}$. Найдём сначала $\sigma(p^\alpha)$:
    $$\sigma(p^\alpha) = 1 + p + p^2 + \dots + p^{\alpha} = \f{p^{\alpha+1}-1}{p-1}$$
    Докажем, что:
    $$
    \sigma(a \cdot b) = \sigma(a) \cdot \sigma(b), \text{ где $(a, b) = 1$}
    $$
    То есть:
    $$\s{d \divides ab}{}d =\s{d_1 \divides a}{}d_1 \cdot \s{d_2 \divides b}{}d_2 $$
    Пусть:
    $$a = p_1^{\alpha_1}\cdot...\cdot p_k^{\alpha_k},\quad b =q_1^{\beta_1}\cdot...\cdot q_s^{\beta_s}$$
    $$d \divides mn \implies d = p_1^{\gamma_1}\cdot...\cdot p_k^{\gamma_k} \cdot q_1^{\omega_1}\cdot...\cdot q_s^{\omega_s} = d_1 \cdot d_2 \implies$$
    $$\implies \s{d \divides ab}{}d =\s{d_1 \divides a}{}d_1 \cdot \s{d_2 \divides b}{}d_2 \implies$$
    $$\implies \sigma(a \cdot b) = \sigma(a) \cdot \sigma(b), \text{ где $(a, b) = 1$}$$
    Получается, что:
    $$\sigma(n) = \sigma(p_1^{\alpha_1}\cdot ... \cdot p_s^{\alpha_s}) = \sigma(p_1^{\alpha_1}) \cdot...\cdot \sigma(p_s^{\alpha_s}) =$$
    $$= \f{p_1^{\alpha_1+1}-1}{p_1-1}\cdot \f{p_1^{\alpha_2+1}-1}{p_2-1} \cdot \dots$$

    \item[\textbf{№4}]Вычислим $\tau(\sigma(108))$:
    $$108 = 3^3\cdot2^2$$
    Следовательно:
    $$\sigma(108) = \f{3^4-1}{3-1}\cdot \f{2^3-1}{2-1} = 40\cdot 7 = 280$$
    $$280 = 5\cdot7\cdot2^3$$
    Получается:
    $$\tau(\sigma(108)) = (1+1)(1+1)(3+1) = 4\cdot4 = 16$$

    \item[\textbf{№5}] Решим уравнения в целых числах:
    \begin{enumerate}
        \item[а)]$19x + 88y = 2$
        $$\mat{
            19 & 88 & | & -2\\
            1 & 0 & | & 0\\
            0 & 1 & | & 0
        }\implies \mat{
            19 & 12 & | & -2\\
            1 & -4 & | & 0\\
            0 & 1 & | & 0
        }\implies \mat{
            7 & 12 & | & -2\\
            5 & -4 & | & 0\\
            -1 & 1 & | & 0
        } \implies$$
        $$\implies \mat{
            7 & 5 & | & -2\\
            5 & -9 & | & 0\\
            -1 & 2 & | & 0
        } \implies 
        \mat{
            2 & 5 & | & -2\\
            14 & -9 & | & 0\\
            -3 & 2 & | & 0
        } \implies \mat{
            2 & 3 & | & -2\\
            14 & -23 & | & 0\\
            -3 & 5 & | & 0
        } \implies$$
        $$\implies \mat{
            2 & 1 & | & -2\\
            14 & -37 & | & 0\\
            -3 & 8 & | & 0
        } \implies \mat{
            0 & 1 & | & -2\\
            88 & -37 & | & 0\\
            -19 & 8 & | & 0
        } \implies \mat{
            1 & 0 &| & -2\\
            -37 & 88&| & 0\\
            8 & -19 &| & 0
        }$$
        $$\implies \mat{
            1 & 0 &| & 0\\
            -37 & 88&| & -74\\
            8 & -19 &| & 16
        }\implies \mat{x \\ y} = \mat{-74\\16} + \mat{88 \\ -19}t \quad \forall t \in \ZZ$$
    
        
        \item[б)]$102x+165y = 9$
        $$\mat{
            102 & 165 & | & -9\\
            1 & 0 & | & 0\\
            0 & 1 & | & 0
        } \implies \mat{
            102 & 63 & | & -9\\
            1 & -1 & | & 0\\
            0 & 1 & | & 0
        } \implies \mat{
            39 & 63 & | & -9\\
            2 & -1 & | & 0\\
            -1 & 1 & | & 0
        } $$$$\implies \mat{
            39 & 24 & | & -9\\
            2 & -3 & | & 0\\
            -1 & 2 & | & 0
        } \implies \mat{
            15 & 24 & | & -9\\
            5 & -3 & | & 0\\
            -3 & 2 & | & 0
        } \implies \mat{
            15 & 9 & | & -9\\
            5 & -8 & | & 0\\
            -3 & 5 & | & 0
        } $$$$\implies \mat{
            6 & 9 & | & -9\\
            13 & -8 & | & 0\\
            -8 & 5 & | & 0
        }\implies \mat{
            6 & 3 & | & -9\\
            13 & -21 & | & 0\\
            -8 & 13 & | & 0
        }\implies \mat{
            0 & 3 & | & -9\\
            55 & -21 & | & 0\\
            -34 & 13 & | & 0
        }$$$$\implies \mat{
            3 & 0 &  | & -9\\
            -21 & 55 &  | & 0\\
            13 & -34 & | & 0
        }\implies \mat{
            3 & 0 &  | & 0\\
            -21 & 55 &  | & -63\\
            13 & -34 & | & 39
        } \implies $$$$\implies \mat{x \\ y} = \mat{-63\\39}+\mat{55\\-34}t \quad \forall t \in \ZZ$$
    \end{enumerate}
\end{enumerate}
\end{document}