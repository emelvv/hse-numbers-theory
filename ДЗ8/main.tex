\documentclass[a4paper]{article}
\usepackage{setspace}
\usepackage[T2A]{fontenc} %
\usepackage[utf8]{inputenc} % подключение русского языка
\usepackage[russian]{babel} %
\usepackage[12pt]{extsizes}
\usepackage{mathtools}
\usepackage{graphicx}
\usepackage{fancyhdr}
\usepackage{amssymb}
\usepackage{amsmath, amsfonts, amssymb, amsthm, mathtools}
\usepackage{tikz}

\usetikzlibrary{positioning}
\setstretch{1.3}

\newcommand{\mat}[1]{\begin{pmatrix} #1 \end{pmatrix}}
\renewcommand{\det}[1]{\begin{vmatrix} #1 \end{vmatrix}}
\renewcommand{\f}[2]{\frac{#1}{#2}}
\newcommand{\dspace}{\space\space}
\newcommand{\s}[2]{\sum\limits_{#1}^{#2}}
\newcommand{\mul}[2]{\prod_{#1}^{#2}}
\newcommand{\sq}[1]{\left[ {#1} \right]}
\newcommand{\gath}[1]{\left[ \begin{array}{@{}l@{}} #1 \end{array} \right.}
\newcommand{\case}[1]{\begin{cases} #1 \end{cases}}
\newcommand{\ts}{\text{\space}}
\newcommand{\lm}[1]{\underset{#1}{\lim}}
\newcommand{\suplm}[1]{\underset{#1}{\overline{\lim}}}
\newcommand{\inflm}[1]{\underset{#1}{\underline{\lim}}}

\renewcommand{\phi}{\varphi}
\newcommand{\lr}{\Leftrightarrow}
\renewcommand{\r}{\Rightarrow}
\newcommand{\rr}{\rightarrow}
\renewcommand{\geq}{\geqslant}
\renewcommand{\leq}{\leqslant}
\newcommand{\RR}{\mathbb{R}}
\newcommand{\CC}{\mathbb{C}}
\newcommand{\QQ}{\mathbb{Q}}
\newcommand{\ZZ}{\mathbb{Z}}
\newcommand{\VV}{\mathbb{V}}
\newcommand{\NN}{\mathbb{N}}
\newcommand{\OO}{\underline{O}}
\newcommand{\oo}{\overline{o}}
\newcommand{\divides}{\;|\;}
\newcommand{\leg}[2]{\left(\f{#1}{#2}\right)}

\DeclarePairedDelimiter\abs{\lvert}{\rvert} %
\makeatletter                               % \abs{}
\let\oldabs\abs                             %
\def\abs{\@ifstar{\oldabs}{\oldabs*}}       %

\begin{document}

\section*{Домашнее задание на 7.03 (Теория чисел)}
 {\large Емельянов Владимир, ПМИ гр №247}\\\\
\begin{enumerate}
    \item[\textbf{№1}]Рассмотрим модуль \( m = 11 \). Тогда функция Эйлера даёт
    \[
    \varphi(11) = 10.
    \]
    Пусть \( g \in \mathbb{Z} \) с \((g,11)=1\). По критерию первообразности, \( g \) является первообразным корнем по модулю 11, если и только если для каждого простого делителя \( q \) числа \( \varphi(11)=10 \) выполняются неравенства:
    \[
    g^{\frac{10}{q}} \not\equiv 1 \pmod{11}.
    \]
    Так как простые делители 10 — это 2 и 5, условие эквивалентно:
    \[
    g^5 \not\equiv 1 \pmod{11} \quad \text{и} \quad g^2 \not\equiv 1 \pmod{11}.
    \]
    
    Переберём все \( g \) из множества \(\{1,2,\dots,10\}\):
    
    \begin{itemize}
        \item \( g = 1 \):
        \[
        1^2 \equiv 1 \pmod{11}, \quad 1^5 \equiv 1 \pmod{11} \quad \Rightarrow \quad \text{не является первообразным.}
        \]
        
        \item \( g = 2 \):
        \[
        2^2 = 4 \not\equiv 1 \pmod{11}, \quad 2^5 = 32 \equiv 10 \not\equiv 
        1 \pmod{11} \quad \]\[\Rightarrow \quad 2 \text{ --- первообразный корень.}
        \]
        
        \item \( g = 3 \):
        \[
        3^2 = 9 \not\equiv 1 \pmod{11}, \quad 3^5 = 243 \equiv 1 \pmod{11} \quad \Rightarrow \quad 3 \text{ не подходит.}
        \]
        
        \item \( g = 4 \):
        \[
        4^5 = 1024 \equiv 1 \pmod{11} \quad \Rightarrow \quad 4 \text{ не подходит.}
        \]
        
        \item \( g = 5 \):
        \[
        5^5 = 3125 \equiv 1 \pmod{11} \quad \Rightarrow \quad 5 \text{ не подходит.}
        \]
        
        \item \( g = 6 \):
        \[
        6^2 = 36 \equiv 3 \not\equiv 1 \pmod{11}, \quad 6^5 \equiv 10 \not\equiv 
        1 \pmod{11} \quad \]\[\Rightarrow \quad 6 \text{ --- первообразный корень.}
        \]
        
        \item \( g = 7 \):
        \[
        7^2 = 49 \equiv 5 \not\equiv 1 \pmod{11}, \quad 7^5 \equiv 10 \not\equiv 
        1 \pmod{11} \quad\]\[ \Rightarrow \quad 7 \text{ --- первообразный корень.}
        \]
        
        \item \( g = 8 \):
        \[
        8^2 = 64 \equiv 9 \not\equiv 1 \pmod{11}, \quad 8^5 = 32768 \equiv 10
         \not\equiv 1 \pmod{11} \quad \]\[\Rightarrow \quad 8 \text{ --- первообразный корень.}
        \]
        
        \item \( g = 9 \):
        \[
        9^5 \equiv 1 \pmod{11} \quad \Rightarrow \quad 9 \text{ не подходит.}
        \]
        
        \item \( g = 10 \):
        \[
        10^2 = 100 \equiv 1 \pmod{11} \quad \Rightarrow \quad 10 \text{ не подходит.}
        \]
    \end{itemize}
    
    Таким образом, первообразными корнями по модулю 11, лежащими на интервале от 0 до 11, являются:
    \[
    \{2,\, 6,\, 7,\, 8\}.
    \]
    

    \item[\textbf{№2}]Пусть $m = 2p+1$, тогда:
    $$\phi(m) = 2p$$
    Значит, чтобы $-2$ был первообразным корнем, нужно, чтобы:
    $$\forall q \divides 2p : (-2)^{\f{2p}{q}} \not\equiv 1 \pmod{2p+1}$$
    
    У числа $2p$ ровно 2 простых делителя $2$ и $p$, рассмотрим каждый:
    \begin{enumerate}
        \item[1)] $q = 2$:
        
        Тогда, нужно, чтобы выполнялось:
        $$(-2)^{p} \not \equiv 1 \pmod{2p+1}$$
        Пусть обратное:
        $$(-2)^{p}  \equiv 1 \pmod{2p+1} \quad (*)$$
        По малой теореме Ферма мы знаем, что:
        $$(-2)^{2p} \equiv 1 \pmod{2p+1}$$
        Значит, чтобы выполнялось $(*)$, нужно:
        $$p = 2p \implies p = 0$$
        Но такого не может быть, так как $p \equiv -1 \pmod{4}$, противоречие. 
        Следовательно $(*)$ не выполняется

        \item[2)] $q = p$:
        Тогда, нужно, чтобы выполнялось:
        $$(-2)^{2} \equiv 4 \not \equiv 1 \pmod{2p+1}$$
        То есть:
        $$4\neq 1 + (2p+1)k, \quad k \in \ZZ$$
        $$3 \neq 2pk+k$$
        Рассмотрим обратное:
        $$3= 2pk+k \implies k \divides 3$$
        Рассмотрим все четыре случая:
        $$\case{
            k = 0: 3= 0 \quad \varnothing\\
            k = 1: 3 = 2p+1 \implies 2p = 2 \implies p = 1\; \varnothing \quad (\text{т.к. } p \equiv -1 \pmod{4}) \\
            k = 2: 1 = 4p \quad \varnothing\\
            k = 3: 3 = 6p+3 \implies p = 0 \quad \varnothing
        }$$
        Следовательно :
        $$(-2)^{2} \equiv 4 \not \equiv 1 \pmod{2p+1}$$
    \end{enumerate}
    Следовательно:
    $$\forall q \divides 2p : (-2)^{\f{2p}{q}} \not\equiv 1 \pmod{2p+1}$$
    А значит $-2$ - первообразный корень

    \item[\textbf{№3}]Модуль \( 79 \) — простое число. Количество первообразных корней по модулю \( 79 \) равно \( \phi(\phi(79)) = \phi(78) \), где \( \phi \) — функция Эйлера. Разложим \( 78 \) на простые множители:
    \[
    78 = 2 \times 3 \times 13
    \]
    Тогда:
    \[
    \phi(78) = \phi(2) \times \phi(3) \times \phi(13) = 1 \times 2 \times 12 = 24
    \]
    Таким образом, существует \( 24 \) первообразных корня по модулю \( 79 \)
    
    Пусть \( g \) — один из первообразных корней по модулю \( 79 \). Тогда все первообразные корни имеют вид:
    \[
    g^k, \quad \text{где } 1 \leq k \leq 77 \text{ и } (k, 78) = 1
    \]
    Произведение всех первообразных корней равно:
    \[
    P = \prod_{\substack{1 \leq k \leq 77 \\ (k, 78) = 1}} g^k
    \]
    Обозначим сумму показателей через:
    \[
    S = \sum_{\substack{1 \leq k \leq 77 \\ (k, 78) = 1}} k
    \]
    Сумма чисел, взаимно простых с \( m \) и меньших \( m \), равна:
    \[
    S = \frac{m \cdot \phi(m)}{2}
    \]
    Для \( m = 78 \):
    \[
    S = \frac{78 \cdot \phi(78)}{2} = \frac{78 \cdot 24}{2} = 78 \cdot 12 = 936
    \]
    Заметим, что \( 936 = 78 \times 12 \), поэтому:
    \[
    S \equiv 0 \pmod{78}
    \]
    Подставляя \( S \) в выражение для \( P \), получаем:
    \[
    P = g^{S} = g^{78 \times 12} = \left(g^{78}\right)^{12}.
    \]
    Поскольку показатель \( g \) равен \( 78 \), выполняется:
    \[
    g^{78} \equiv 1 \pmod{79}
    \]
    Следовательно:
    \[
    P \equiv 1^{12} \equiv 1 \pmod{79}
    \]\\

    \item[\textbf{№4}]
    Пусть \( g \) — первообразный корень по модулю \( m \), и \( k \in \mathbb{N} \). Требуется доказать, что:
    \[
    \text{ord}_m(g^k) = \frac{\varphi(m)}{(k, \varphi(m))}
    \]
    показатель элемента \( g^k \) по модулю \( m \) — это наименьшее натуральное число \( d \), такое что:
    \[
    (g^k)^d \equiv 1 \pmod{m}.
    \]
    Это эквивалентно:
    \[
    g^{kd} \equiv 1 \pmod{m}.
    \]
    Так как \( g \) — первообразный корень, его показатель равен \( \varphi(m) \). Следовательно:
    \[
    g^{\varphi(m)} \equiv 1 \pmod{m},
    \]
    и для любого \( t \in \mathbb{N} \):
    \[
    g^t \equiv 1 \pmod{m} \iff \varphi(m) \mid t
    \]
    Из условия \( g^{kd} \equiv 1 \pmod{m} \) следует, что:
    \[
    \varphi(m) \mid kd.
    \]
    Минимальное \( d \), удовлетворяющее \( \varphi(m) \mid kd \), определяется как:
    \[
    d = \frac{\varphi(m)}{(k, \varphi(m))}.
    \]
    Это следует из того, что:
    \[
    \text{НОК}(k, \varphi(m)) = \frac{k \cdot \varphi(m)}{(k, \varphi(m))},
    \]
    и тогда:
    \[
    d = \frac{\text{НОК}(k, \varphi(m))}{k} = \frac{\varphi(m)}{(k, \varphi(m))}.
    \]
    Предположим, существует \( d' < d \), такое что \( \varphi(m) \mid kd' \). Тогда:
    \[
    kd' = \varphi(m) \cdot \frac{k}{\gcd(k, \varphi(m))} \cdot \frac{d'}{d}.
    \]
    Но \( \frac{d'}{d} < 1 \), что противоречит целочисленности. Следовательно, \( d \) действительно минимально.

    Поэтому:
    $$\text{ord}_m(g^k) = \frac{\varphi(m)}{(k, \varphi(m))}$$

    \item[\textbf{№5}]Количество элементов в приведённой системе вычетов по модулю $p$: 
    \[\phi(p)= p - 1 = 2^k \]
    Любой элемент \( g \) этой системы имеет показатель \( d \), делящий \( 2^k \), то есть \( d = 2^m \), где \( 0 \leq m \leq k \). Первообразный корень — это элемент показателя \( 2^k \).

    \begin{itemize}
        \item \textbf{Если \( g \) — первообразный корень},
         то его показатель \( 2^k \). Если бы \( g \) был квадратичным вычетом, 
         то существовал бы элемент \( x \), такой что \( g \equiv x^2 \mod p \).
          Тогда показатель \( x \) был бы \( 2^{k + 1} \) ($g^{2^k} \equiv x^2$), что невозможно, 
          так как максимальный показатель \( 2^k \).
           Следовательно, \( g \) — квадратичный невычет.
        
        \item \textbf{Если \( g \) — квадратичный невычет}. 
        Предположим, что показатель \( g \) равен \( 2^m \), 
        где \( m < k \). Тогда \( g^{2^{m - 1}} \equiv -1 \mod p \).
         Возведём обе части в квадрат:
        \[
        g^{2^{m}} \equiv 1 \mod p.
        \]
        Это означает, что \( g^{2^{m}} \equiv 1 \mod p \), 
        что противоречит тому, что показатель \( g \) равен \( 2^m \). 
        Следовательно, \( m = k \), и \( g \) — первообразный корень.
    \end{itemize}
    Первообразный корень имеет максимальный показатель \( 2^k \), что исключает возможность быть квадратом. Квадратичный невычет, не будучи квадратом, обязан иметь максимальный показатель. Таким образом, эквивалентность доказана.


\end{enumerate}
\end{document}