\documentclass[a4paper]{article}
\usepackage{setspace}
\usepackage[T2A]{fontenc} %
\usepackage[utf8]{inputenc} % подключение русского языка
\usepackage[russian]{babel} %
\usepackage[12pt]{extsizes}
\usepackage{mathtools}
\usepackage{graphicx}
\usepackage{fancyhdr}
\usepackage{amssymb}
\usepackage{amsmath, amsfonts, amssymb, amsthm, mathtools}
\usepackage{tikz}

\usetikzlibrary{positioning}
\setstretch{1.3}

\newcommand{\mat}[1]{\begin{pmatrix} #1 \end{pmatrix}}
\renewcommand{\det}[1]{\begin{vmatrix} #1 \end{vmatrix}}
\renewcommand{\f}[2]{\frac{#1}{#2}}
\newcommand{\dspace}{\space\space}
\newcommand{\s}[2]{\sum\limits_{#1}^{#2}}
\newcommand{\mul}[2]{\prod_{#1}^{#2}}
\newcommand{\sq}[1]{\left[ {#1} \right]}
\newcommand{\gath}[1]{\left[ \begin{array}{@{}l@{}} #1 \end{array} \right.}
\newcommand{\case}[1]{\begin{cases} #1 \end{cases}}
\newcommand{\ts}{\text{\space}}
\newcommand{\lm}[1]{\underset{#1}{\lim}}
\newcommand{\suplm}[1]{\underset{#1}{\overline{\lim}}}
\newcommand{\inflm}[1]{\underset{#1}{\underline{\lim}}}

\renewcommand{\phi}{\varphi}
\newcommand{\lr}{\Leftrightarrow}
\renewcommand{\r}{\Rightarrow}
\newcommand{\rr}{\rightarrow}
\renewcommand{\geq}{\geqslant}
\renewcommand{\leq}{\leqslant}
\newcommand{\RR}{\mathbb{R}}
\newcommand{\CC}{\mathbb{C}}
\newcommand{\QQ}{\mathbb{Q}}
\newcommand{\ZZ}{\mathbb{Z}}
\newcommand{\VV}{\mathbb{V}}
\newcommand{\NN}{\mathbb{N}}
\newcommand{\OO}{\underline{O}}
\newcommand{\oo}{\overline{o}}
\newcommand{\divides}{\;|\;}
\newcommand{\leg}[2]{\left(\f{#1}{#2}\right)}

\DeclarePairedDelimiter\abs{\lvert}{\rvert} %
\makeatletter                               % \abs{}
\let\oldabs\abs                             %
\def\abs{\@ifstar{\oldabs}{\oldabs*}}       %

\begin{document}

\section*{Домашнее задание на 21.02 (Теория чисел)}
 {\large Емельянов Владимир, ПМИ гр №247}\\\\
\begin{enumerate}
    \item[\textbf{№1}]$x^2 \equiv 219 \pmod{383}$2
    Найдём символ Лежандра:
    $$\leg{219}{383} = \leg{383}{219} \cdot (-1)^{\f{218\cdot 382}{4}} = -\leg{383}{219} =- \leg{64}{219} = -\leg{2^6}{219} = -\leg{2}{219}^6 $$
    $$=-\left((-1)^{\f{219^2-1}{8}}\right)^6 = -(-1)^{219^2-1} = -1$$
    \textbf{Ответ: } неразрешимо

    \item[\textbf{№2}]Рассмотрим:
    $$\leg{5}{p} = (-1)^{\f{4(p-1)}{4}}\leg{p}{5} = \leg{p}{5}$$
    При $p = \pm 1 \pmod{5}$:
    $$\leg{p}{5} = 1$$
    При $p = \pm 2 \pmod{5}$:
    $$\leg{p}{5} = -1$$
    Получаем, что:
    $$\leg{5}{p} = \case{
        1 , \text{ при } p \equiv \pm 1 \pmod{5}\\
        -1, \text{ при } p \equiv \pm 2 \pmod{5}
    }$$

    \item[\textbf{№3}]Модуль \(128 \cdot 151 \cdot 199\) раскладывается на взаимно
     простые множители: 
     $$\text{\(2^7\), \(151\), \(199\)}$$ 
     Получаем систему:
     $$\case{
        x^2+2x+72=0 \pmod{2^7}\\
        x^2+2x+72=0 \pmod{151}\\
        x^2+2x+72=0 \pmod{199}
     }\lr \case{
        (x+1)^2=-71 \pmod{2^7}\\
        (x+1)^2=-71 \pmod{151}\\
        (x+1)^2=-71 \pmod{199}
     }
     $$
    У первого сравнения, очевидно, 2 решения, а для остальных найдём символы Лежандра:
    $$\leg{-71}{151}=\leg{80}{151}=\leg{2^4\cdot 5}{151} = \leg{2}{151}^4\cdot \leg{5}{151} = $$
    $$=\leg{2}{151}^4\cdot \leg{5}{151} = \leg{5}{151} = \leg{151}{5}(-1)^{\f{150\cdot 4}{4}} = 1$$
    
    $$\leg{-71}{199} = \leg{128}{199} = \leg{2^7}{199} = \leg{2}{199} = 1$$
    Следовательно, у второго и третьего сравнения по 2 решения. Итоговое количество решений:
    $$2 \times 2 \times 2 = 8$$
    \textbf{Ответ: } $8$

    \item[\textbf{№4}]Чтобы доказать, что для простого числа Ферма 
    \( f_n = 2^{2^n} + 1 \) выполняется сравнение 
    \( 3^{(f_n-1)/2} \equiv -1 \pmod{f_n} \),
     воспользуемся критерием Эйлера.
        
    Для простого \( p \) и целого \( a \), не делящегося на \( p \), верно:
    \[
    a^{(p-1)/2} \equiv \left( \frac{a}{p} \right) \pmod{p},
    \]
    где \( \left( \frac{a}{p} \right) \) — символ Лежандра.
    Подставим $a = 3$ и $p = f_n$:
    \[
    3^{(f_n-1)/2} \equiv \leg{3}{f_n}\pmod{f_n}
    \]
    Докажем, что $\leg{3}{f_n} = -1$
    $$\leg{3}{f_n} = \leg{f_n}{3} (-1)^{\f{f_n-1}{2}} = \leg{2^{2^n}+1}{3} (-1)^{2^{2^n-1}} =
     \leg{2^{2^n}+1}{3} = $$
     $$=\leg{(-1)^{2^n}+1}{3} = \leg{-1}{3} = -1$$
    
     Следовательно:
     \[
        3^{\f{f_n-1}{2}} \equiv -1\pmod{f_n}
    \]
    ч.т.д.

    \item[\textbf{№5}]
    Пусть \( p \) и \( q = 2p + 1 \) — простые числа, причём 
    \( p \equiv 3 \pmod{4} \). Требуется доказать, что число Мерсенна 
    \( M_p = 2^p - 1 \) простое только при \( p = 3 \).
    
    Подставим \( p = 3 \):  
    \[ q = 2 \cdot 3 + 1 = 7 \text{ (простое)}\] 
    \[ M_3 = 2^3 - 1 = 8 - 1 = 7 \text{ (простое)}\]
    Условие выполнено.

    Покажем, что для \( p > 3 \) число \( M_p \) составное:
    
    Так как \( q = 2p + 1 \) — простое, применим малую теорему Ферма к \( 2 \) по модулю \( q \):  
    \[
    2^{q-1} \equiv 1 \pmod{q}.
    \]  
    Поскольку \( q - 1 = 2p \), получаем:  
    \[
    2^{2p} \equiv 1 \pmod{q}.
    \]  
    Это означает, что \( 2^p \equiv \pm 1 \pmod{q} \).  

    \begin{enumerate}
        \item[1)]
        Если \( 2^p \equiv 1 \pmod{q} \):
    
        Тогда \( 2^p - 1 \equiv 0 \pmod{q} \), то есть \( q \) делит \( M_p \).  
    
        Так как \( q = 2p + 1 \) и \( p > 3 \), то \( q < M_p \), следовательно, \( M_p \) составное.  
        
        \item[2)]Если \( 2^p \equiv -1 \pmod{q} \):
        \item[] 
        Возведём в квадрат:  
        \[
        (2^p)^2 = 2^{2p} \equiv 1 \pmod{q}.
        \]  
        Это совпадает с малой теоремой Ферма, но для \( p \equiv 3 \pmod{4} \) выполняется следующее: 
        \begin{enumerate}
            \item[$\cdot$]\( p = 4k + 3 \), тогда \( q = 2(4k + 3) + 1 = 8k + 7 \)
            \item[$\cdot$] Число \( 2 \) является квадратичным вычетом по модулю \( q \), так как \( q \equiv 7 \pmod{8} \) (по теореме с лекции). 
            По критерию Эйлера:  
            \[
            2^{\frac{q-1}{2}} \equiv 1 \pmod{q}.
            \]  
            Подставляя \( \frac{q-1}{2} = 4k + 3 \), получаем:  
            \[
            2^{4k + 3} \equiv 1 \pmod{q}.
            \]  
            Но \( 2^{4k + 3} = 2^p \), поэтому \( 2^p \equiv 1 \pmod{q} \).  
            Это сводится к первому случаю, где \( q \) делит \( M_p \), делая его составным    
        \end{enumerate}

        Следовательно, $M_p$ простое только при $p = 3$
    \end{enumerate}
\end{enumerate}
\end{document}