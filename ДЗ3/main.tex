\documentclass[a4paper]{article}
\usepackage{setspace}
\usepackage[T2A]{fontenc} %
\usepackage[utf8]{inputenc} % подключение русского языка
\usepackage[russian]{babel} %
\usepackage[12pt]{extsizes}
\usepackage{mathtools}
\usepackage{graphicx}
\usepackage{fancyhdr}
\usepackage{amssymb}
\usepackage{amsmath, amsfonts, amssymb, amsthm, mathtools}
\usepackage{tikz}

\usetikzlibrary{positioning}
\setstretch{1.3}

\newcommand{\mat}[1]{\begin{pmatrix} #1 \end{pmatrix}}
\renewcommand{\det}[1]{\begin{vmatrix} #1 \end{vmatrix}}
\renewcommand{\f}[2]{\frac{#1}{#2}}
\newcommand{\dspace}{\space\space}
\newcommand{\s}[2]{\sum\limits_{#1}^{#2}}
\newcommand{\mul}[2]{\prod_{#1}^{#2}}
\newcommand{\sq}[1]{\left[ {#1} \right]}
\newcommand{\gath}[1]{\left[ \begin{array}{@{}l@{}} #1 \end{array} \right.}
\newcommand{\case}[1]{\begin{cases} #1 \end{cases}}
\newcommand{\ts}{\text{\space}}
\newcommand{\lm}[1]{\underset{#1}{\lim}}
\newcommand{\suplm}[1]{\underset{#1}{\overline{\lim}}}
\newcommand{\inflm}[1]{\underset{#1}{\underline{\lim}}}

\renewcommand{\phi}{\varphi}
\newcommand{\lr}{\Leftrightarrow}
\renewcommand{\r}{\Rightarrow}
\newcommand{\rr}{\rightarrow}
\renewcommand{\geq}{\geqslant}
\renewcommand{\leq}{\leqslant}
\newcommand{\RR}{\mathbb{R}}
\newcommand{\CC}{\mathbb{C}}
\newcommand{\QQ}{\mathbb{Q}}
\newcommand{\ZZ}{\mathbb{Z}}
\newcommand{\VV}{\mathbb{V}}
\newcommand{\NN}{\mathbb{N}}
\newcommand{\OO}{\underline{O}}
\newcommand{\oo}{\overline{o}}
\newcommand{\divides}{\;|\;}

\DeclarePairedDelimiter\abs{\lvert}{\rvert} %
\makeatletter                               % \abs{}
\let\oldabs\abs                             %
\def\abs{\@ifstar{\oldabs}{\oldabs*}}       %

\begin{document}

\section*{Домашнее задание на 31.01 (Теория чисел)}
 {\large Емельянов Владимир, ПМИ гр №247}\\\\
\begin{enumerate}
    \item[\textbf{№1}]Пусть стороны треугольника $a, b, c \in \ZZ$. Выполняется т.Пифагора:
    $$a^2 + b^2 = c^2$$

    \item[\textbf{№2}]Решим сравнения:
    \begin{enumerate}
        \item[a)] $19x \equiv 2 (mod 88)$
        
        Это равносильно:
        $$19x-2 = 88y \implies 19x-88y=2$$
        Решим:
        $$\mat{
            19 & -88 & | & -2 \\
            1 & 0 & | & 0 \\
            0 & 1 & | & 0 \\

        } \implies\mat{
            19 & 7 & | & -2 \\
            1 & 5 & | & 0 \\
            0 & 1 & | & 0 \\

        } \implies \mat{
            -2 & 7 & | & -2 \\
            -14 & 5 & | & 0 \\
            -3 & 1 & | & 0 \\

        }$$$$\implies \mat{
            2 & 1 & | & -2 \\
            14 & -37 & | & 0 \\
            3 & -8 & | & 0 \\

        }\implies \mat{
            1 & 0 & | & -2 \\
            -37 & 88 & | & 0 \\
            -8 & 19 & | & 0 \\

        }\implies \mat{
            1 & 0 & | & 0 \\
            -37 & 88 & | & -74 \\
            -8 & 19 & | & 16 \\

        }$$
        $$\implies \mat{x\\y} = \mat{88\\19}t + \mat{-74\\16} \implies x = 88t-74 \quad t\in \ZZ$$
    
        \item[б)]$102x \equiv 9 (mod 165)$
        Это эквивалентно:
        $$102x - 165y = 9$$
        Решим:
        $$\mat{
            102 & -165 & | & -9 \\
            1 & 0 & | & 0 \\
            0 & 1 & | & 0 \\

        } \implies \mat{
            -24 & 3 & | & -9 \\
            3 & -21 & | & 0 \\
            -2 & 15 & | & 0 \\
        }\implies \mat{
            0 & 3 & | & -9 \\
            -165 & -21 & | & 0 \\
            118 & 15 & | & 0 \\
        }$$$$\implies \mat{
            3 & 0 & | & 0 \\
            -21 & -165 & | & -63 \\
            15 & 118 & | & 45 \\
        }\implies \mat{x\\y} = \mat{-165\\118}t + \mat{-63\\45} \implies$$
        $$\implies x = -165t - 63 \quad t\in \ZZ$$
    \end{enumerate}

    \item[\textbf{№3}]По теореме Ферма:
    $$5^{p-1} = 1 (mod \; p)$$
    При этом:
    $$5^{p^2} = 1 (mod \; p)$$
    Значит нужно, чтобы:
    $$p-1 \divides p^2 \implies p^2 \equiv 0 (mod \; p-1)$$
    Т.к. $p$ и $p-1$ взаимно просты, то можно сократить на $p$:
    $$p \equiv 0 (mod \; p-1) \implies 1 = (p-1)t $$
    При $t \neq 0$:
    $$ p = \f{1+t}{t} = \f{1}{t}+1 \implies t = 1, -1 \implies p = 2, 0$$
    При $t = 0$:
    $$1 = 0 \quad \emptyset$$
    
    \textbf{Ответ:} $0$ и $2$

    \item[\textbf{№4}]Чтобы найти все основания $ a $, для которых число $ n = 15 $ является псевдопростым, мы должны проверить условие:

    $$
    a^{n-1} \equiv 1 \pmod{n}
    $$
    
    где $ n - 1 = 14 $. Таким образом, нам нужно проверить:
    
    $$
    a^{14} \equiv 1 \pmod{15}
    $$
    
    Сначала найдем все числа $ a $, такие что $ (a, 15) = 1 $. Число 15 имеет делители 3 и 5, поэтому $ a $ не должно быть кратно 3 или 5. Таким образом, возможные значения $ a $ в пределах от 1 до 14:

    $$
    a = 1, 2, 4, 7, 8, 11, 13, 14
    $$
    Условие $$
    a^{14} \equiv 1 \pmod{15}
    $$
    По теореме Эйлера эквивалентно (т.к. $\phi(15) = 8$, значит $a^{8} = 1 \pmod{15}$):
    $$
    a^{6} \equiv 1 \pmod{15}
    $$

    Теперь нам нужно проверить каждое значение $ a $ из списка $ a = 2, 4, 7, 8, 11, 13, 14 $ на выполнение условия $ a^{6} \equiv 1 \pmod{15} $.
    
    1. Для $ a = 2 $:
    $$
    2^{6} = 64 \quad \Rightarrow \quad 64 \mod 15 = 4 \quad \Rightarrow \quad 2^{6} \equiv 4 \pmod{15} \quad (\text{не подходит}).
    $$

    2. Для $ a = 4 $:
    $$
    4^{6} = (4^{2})^{3} = 16^{3} \equiv 1^{3} \equiv 1 \pmod{15} \quad (\text{подходит}).
    $$

    3. Для $ a = 7 $:
    $$
    7^{6} = (7^{2})^{3} = 49^{3} \equiv 4^{3} \pmod{15}.
    $$
    Теперь вычислим $ 4^{3} $:
    $$
    4^{3} = 64 \quad \Rightarrow \quad 64 \mod 15 = 4 \quad \Rightarrow \quad 7^{6} \equiv 4 \pmod{15} \quad (\text{не подходит}).
    $$

    4. Для $ a = 8 $:
    $$
    8^{6} = (8^{2})^{3} = 64^{3} \equiv 4^{3} \pmod{15}.
    $$
    Как и в предыдущем случае:
    $$
    4^{3} = 64 \quad \Rightarrow \quad 64 \mod 15 = 4 \quad \Rightarrow \quad 8^{6} \equiv 4 \pmod{15} \quad (\text{не подходит}).
    $$

    5. Для $ a = 11 $:
    $$
    11^{6} = (11^{2})^{3} = 121^{3} \equiv 1^{3} \equiv 1 \pmod{15} \quad (\text{подходит}).
    $$

    6. Для $ a = 13 $:
    $$
    13^{6} = (13^{2})^{3} = 169^{3} \equiv 4^{3} \pmod{15}.
    $$
    Как и ранее:
    $$
    4^{3} = 64 \quad \Rightarrow \quad 64 \mod 15 = 4 \quad \Rightarrow \quad 13^{6} \equiv 4 \pmod{15} \quad (\text{не подходит}).
    $$

    7. Для $ a = 14 $:
    $$
    14^{6} = (-1)^{6} \equiv 1 \pmod{15} \quad (\text{подходит}).
    $$
    
    \textbf{Ответ:} $4, 11, 14$

    \item[\textbf{№5}]$\case{
        x \equiv 2 \pmod{11}\\ 
        x \equiv 1 \pmod{13}
    }$
    Это равносильно:
    $$\case{
        x-2 = 11k\\
        x-1 = 13p\\
    }, \quad k, p \in \ZZ$$
    Следовательно,
    $$x = 11k+2 = 13p+1 \implies 11k-13p = -1$$
    Решим диафантово уравнение:
    $$\mat{
        11 & -13 & | & 1\\ 
        1 & 0 & | & 0\\
        0 & 1 & | & 0\\
    } \implies \mat{
        -2 & -13 & | & 1\\ 
        1 & 0 & | & 0\\
        1 & 1 & | & 0\\
    }\implies \mat{
        -2 & 1& | & 1\\ 
        1 & -7 & | & 0\\
        1 & -6 & | & 0\\
    }
    $$
    $$\implies \mat{
        0 & 1& | & 1\\ 
        -13 & -7 & | & 0\\
        -11 & -6 & | & 0\\
    }\implies \mat{
        1 & 0 & | & 0\\ 
        -7 & -13 & | & 7\\
        -6 & -11 & | & 6\\
    } \implies \mat{k \\ p} = \mat{-13 \\ -11}t + \mat{7\\6}$$
    Получаем:
    $$\case{
        x = 11\cdot(-13t+7)+2\\
        x = 13\cdot(-11t+6)+1
    }\implies \case{
        x = -143t+79\\
        x = -143t+79
    } \implies x = -143t+79 \quad t \in \ZZ$$

    \textbf{Ответ: }$x = -143t+79 \quad t \in \ZZ \lr x \equiv 79 \pmod{-143}$
    

\end{enumerate}
\end{document}