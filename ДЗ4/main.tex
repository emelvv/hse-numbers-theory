\documentclass[a4paper]{article}
\usepackage{setspace}
\usepackage[T2A]{fontenc} %
\usepackage[utf8]{inputenc} % подключение русского языка
\usepackage[russian]{babel} %
\usepackage[12pt]{extsizes}
\usepackage{mathtools}
\usepackage{graphicx}
\usepackage{fancyhdr}
\usepackage{amssymb}
\usepackage{amsmath, amsfonts, amssymb, amsthm, mathtools}
\usepackage{tikz}

\usetikzlibrary{positioning}
\setstretch{1.3}

\newcommand{\mat}[1]{\begin{pmatrix} #1 \end{pmatrix}}
\renewcommand{\det}[1]{\begin{vmatrix} #1 \end{vmatrix}}
\renewcommand{\f}[2]{\frac{#1}{#2}}
\newcommand{\dspace}{\space\space}
\newcommand{\s}[2]{\sum\limits_{#1}^{#2}}
\newcommand{\mul}[2]{\prod_{#1}^{#2}}
\newcommand{\sq}[1]{\left[ {#1} \right]}
\newcommand{\gath}[1]{\left[ \begin{array}{@{}l@{}} #1 \end{array} \right.}
\newcommand{\case}[1]{\begin{cases} #1 \end{cases}}
\newcommand{\ts}{\text{\space}}
\newcommand{\lm}[1]{\underset{#1}{\lim}}
\newcommand{\suplm}[1]{\underset{#1}{\overline{\lim}}}
\newcommand{\inflm}[1]{\underset{#1}{\underline{\lim}}}

\renewcommand{\phi}{\varphi}
\newcommand{\lr}{\Leftrightarrow}
\renewcommand{\r}{\Rightarrow}
\newcommand{\rr}{\rightarrow}
\renewcommand{\geq}{\geqslant}
\renewcommand{\leq}{\leqslant}
\newcommand{\RR}{\mathbb{R}}
\newcommand{\CC}{\mathbb{C}}
\newcommand{\QQ}{\mathbb{Q}}
\newcommand{\ZZ}{\mathbb{Z}}
\newcommand{\VV}{\mathbb{V}}
\newcommand{\NN}{\mathbb{N}}
\newcommand{\OO}{\underline{O}}
\newcommand{\oo}{\overline{o}}
\newcommand{\divides}{\;|\;}

\DeclarePairedDelimiter\abs{\lvert}{\rvert} %
\makeatletter                               % \abs{}
\let\oldabs\abs                             %
\def\abs{\@ifstar{\oldabs}{\oldabs*}}       %

\begin{document}

\section*{Домашнее задание на 07.02 (Теория чисел)}
 {\large Емельянов Владимир, ПМИ гр №247}\\\\
\begin{enumerate}
    \item[\textbf{№1}]Доказательство:
    \begin{enumerate}
        \item[$\r$] Пусть $p$ - простое, тогда по т.Вильсона:
        $$(p-1)! \equiv -1 \pmod{p}$$
        Но мы знаем, что:
        $$p-1 \equiv -1 \pmod{p}$$
        А значит:
        $$(p-2)! \equiv 1 \pmod{p}$$

        \item[$\Leftarrow$] Пусть:
        $$(p-2)! \equiv 1 \pmod{p}$$
        докажем, что $p$ - простое.

        Пусть $p$ - непростое, тогда существуют такие $a$ и $b$, что:
        $$p = ab$$
        А значит:
        $$(p-2)! =(p-2)(p-3)\dots a \dots b \dots 2\cdot 1$$
        Поэтому:
        $$p=ab \divides (p-2)!$$
        Следовательно:
        $$(p-2)! \equiv 0 \pmod{p}$$
        Но по условию:
        $$(p-2)! \equiv 1 \pmod{p}$$
        Противоречие, значит $p$ - простое.
    \end{enumerate}

    \item[\textbf{№2}]Чтобы решить систему
    $$\case{
        x \equiv 4 \pmod{15}\\
        x \equiv -1 \pmod{16}\\
        x \equiv 11 \pmod{17}
    }$$
    воспользуемся к.т.о. . Единственное решение:
    $$x = \s{i=1}{k}M_i b_i \pmod{M}$$
    Где:
    $$M_i b_i = a_i \pmod{m_i}$$
    $$M_1 = 16\cdot 17 = 272 \equiv 2 \pmod{15}$$
    $$M_2 = 15\cdot 17 = 255 \equiv -1 \pmod{16}$$
    $$M_3 = 15\cdot 16 = 240 \equiv 2 \pmod{17}$$
    $$M_1b_1 \equiv a_1 \pmod{m_1} \lr 2 b_1 \equiv 4 \pmod{15} \r b_1 = 2$$
    $$M_2b_2 \equiv a_2 \pmod{m_2} \lr -b_2 \equiv -1 \pmod{16} \r b_2 = 1$$
    $$M_3b_3 \equiv a_3 \pmod{m_3} \lr 2b_3 \equiv 11 \equiv -6 \pmod{17} \r b_3 = -3$$
    Найдём решение:
    $$M = 15\cdot 16 \cdot 17 = 4080$$
    $$x = 272\cdot 2 +255 \cdot 1+240\cdot (-3) = 79 \pmod{4080}$$
    \textbf{Ответ: } $79 \pmod{4080}$\\

    \item[\textbf{№3}]\begin{enumerate}
        \item[a)]$x^2-1 \equiv 0 \pmod{15}$
        Получаем:
        $$(x-1)(x+1)\equiv 0 \pmod{15}$$
        Разберём все возможные случаи:
        $$\gath{
            \case{
                x-1 \equiv 0 \pmod{3}\\
                x+1 \equiv 0 \pmod{5}\\
            } \implies x = -11 \pmod{30}\\
            \case{
                x-1 \equiv 0 \pmod{5}\\
                x+1 \equiv 0 \pmod{3}\\
            } \implies x = 11 \pmod{30}\\
            x-1 \equiv 0 \pmod{15} \implies x = 1 \pmod{15}\\
            x+1 \equiv 0 \pmod{15} \implies x = -1 \pmod{15}
        }$$
        \textbf{Ответ: }$\gath{
            x = 1 \pmod{15} \\
            x = -1 \pmod{15} \\
            x = -11 \pmod{30}\\
            x = 11 \pmod{30}
         }$ 

        \item[б)]$x^2-1 \equiv 0 \pmod{56}$
        $$56 = 8\cdot 7$$
        Получаем:
        $$(x-1)(x+1)\equiv 0 \pmod{56}$$
        Так $x-1$ и $x+1$ одновременно либо чётные, либо нечётные, то для искомого сравнения возможны только случаи:
        $$
        \case{
            7\cdot 2 \divides x-1\\
            2^2 \divides x+1
        }\\
        \case{
            7\cdot 2^2 \divides x-1\\
            2 \divides x+1
        }
        \case{
            7\cdot 2 \divides x+1\\
            2^2 \divides x-1
        }\\
        \case{
            7\cdot 2^2 \divides x+1\\
            2 \divides x-1
        }
        $$
        Или когда:
        $$56 \divides x-1 \text{ или } 56 \divides x+1$$\\
    \end{enumerate}

    \item[\textbf{№4}]$\varphi(4^x6^y) = 2\varphi(35^z)$
    Это эквивалентно:
    $$2^{2x+y}3^y(1-\f{1}{2})(1-\f{1}{3}) = 2\cdot 35^z (1 - \f{1}{5})(1-\f{1}{7})$$
    $$2^{2x+y}3^y\f{1}{2}\f{2}{3} = 2\cdot 35^z \f{4}{5}\f{6}{7}$$
    $$2^{2x+y}3^{y-1} =  48 \cdot 35^{z-1} = 2^{4}\cdot 3 \cdot 5^{z-1}\cdot 7^{z-1}$$
    Следовательно:
    $$\case{
        2x+y = 4\\
        y-1 = 1\\
        z = 1
    } \implies \case{
        x = 1\\
        y = 2\\
        z = 1
    } $$

    \textbf{Ответ: }$(x, y, z) = (1, 2, 1)$\\

    \item[\textbf{№5}]Чтобы доказать, что отображение 
    $$
    \operatorname{Enc}_{e}(\bar{a})=\overline{a^{e}}
    $$
    взаимно однозначно отображает $\mathbb{Z}_{n}^{*}$ на себя, нам нужно показать, что оно является биекцией, то есть, что оно инъективно и сюръективно.
    
    Для инъективности нам нужно показать, что если 
    $$
    \operatorname{Enc}_{e}(\bar{a_1}) = \operatorname{Enc}_{e}(\bar{a_2}),
    $$
    то $\bar{a_1} = \bar{a_2}$.

    Предположим, что 
    $$
    \overline{a_1^e} = \overline{a_2^e}.
    $$
    Это означает, что 
    $$
    a_1^e \equiv a_2^e \mod n.
    $$
    Так как $(e, \varphi(n)) = 1$, существует обратный элемент $d$ по модулю $\varphi(n)$, такой что 
    $$
    ed \equiv 1 \mod \varphi(n).
    $$
    Теперь возьмем обе стороны уравнения $a_1^e \equiv a_2^e \mod n$ и возведем в степень $d$:
    $$
    (a_1^e)^d \equiv (a_2^e)^d \mod n.
    $$
    $$
    a_1^{ed} \equiv a_2^{ed} \mod n.
    $$
    Так как $ed \equiv 1 \mod \varphi(n)$, мы можем записать:
    $$
    a_1 \equiv a_2 \mod n.
    $$
    Таким образом, $\bar{a_1} = \bar{a_2}$, что доказывает инъективность.

    Теперь покажем, что отображение сюръективно. Для этого нужно показать, что для любого $\bar{b} \in \mathbb{Z}_{n}^{*}$ существует $\bar{a} \in \mathbb{Z}_{n}^{*}$, такое что 
    $$
    \operatorname{Enc}_{e}(\bar{a}) = \bar{b}.
    $$
    Пусть $\bar{b} \in \mathbb{Z}_{n}^{*}$. Поскольку $(e, \varphi(n)) = 1$, существует $d$ такое, что 
    $$
    ed \equiv 1 \mod \varphi(n).$$
    Теперь мы можем взять $\bar{a} = \overline{b^d}$. Тогда:
    $$
    \operatorname{Enc}_{e}(\bar{a}) = \overline{(b^d)^e} = \overline{b^{de}}.
    $$
    Так как $de \equiv 1 \mod \varphi(n)$, мы имеем:
    $$
    b^{de} \equiv b \mod n.
    $$
    Следовательно,
    $$
    \operatorname{Enc}_{e}(\bar{a}) = \overline{b}.
    $$
    Таким образом, для любого $\bar{b} \in \mathbb{Z}_{n}^{*}$ существует $\bar{a} \in \mathbb{Z}_{n}^{*}$, такое что $\operatorname{Enc}_{e}(\bar{a}) = \bar{b}$, что доказывает сюръективность.

    Следовательно, искомое отображение - биекция.


\end{enumerate}
\end{document}